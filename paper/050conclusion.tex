\noindent We have proposed a novel way of generating aggregate neighborhood features in an unsupervised manner from unlabelled data of nodes adjacent to a user in the network's social graph. Apart from data collection efforts these features are free and lead to more accurate classification results. 

We conducted experiments on the expanded \textsc{Cresci-2018} dataset. In order to generate the neighborhood features we have collected a novel dataset of 4.6 million Twitter user profiles based on the users followed by or following users in the \textsc{Cresci-2018} dataset. In our analysis of the retrieved data we have found a number of egograph features and aggregate features across user neighborhoods that can help in bot classification. We found the most valuable neighborhood features to be the ones based on a node's predecessors. Specifically median out-degree of predecessors, median favourites of predecessors, median status count of predecessors, median account age of predecessors, median favourites of successors and the egograph density and reciprocity seemed to be most effective on this dataset. 

We compared our method to different baseline models. We focus on the best performing baseline classifiers, namely the random forest and neural network classifiers. By adding these neigborhood- and egograph-based features to the data the classification algorithm is trained on, we can improve classification performance. Our method outperforms the baseline on the \textsc{Cresci-2018} dataset. However, as we believe this method to have potential for bigger improvements over the baseline than shown in this work, we plan to test our method on higher quality data in the future.

Lastly, we do note that our method has its limitations as it only focuses on classifying users based on profile and social graph features. When it comes to classification of accounts that behave suspiciously, our method may not be effective on its own, but we believe in can be used in conjunction with existing methods to improve overall performance.