We trained a classification model to classify users as $l \in \{\text{human},\text{bot}\}$ using the following features:

\begin{multicols}{2}
\begin{itemize}
    \item embedding vector username
    \item embedding vector screen name
    \item number of “followers” (in-degree)
    \item number of “following” (out-degree)
    \item number of times the user is listed
    \item number of favorites
    \item number of status
    \item date created
    \item default profile (boolean)
    \item default profile image (boolean)
    \item reputation
    \item successor/predecessor reputation
    \item successor/predecessor in-degree
    \item successor/predecessor out-degree
    \item successor/predecessor favorites count
    \item successor/predecessor status count
    \item successor/predecessor listed count
    \item successor/predecessor account age
    \item clustering coefficient
    \item eigenvector centrality
    \item density
\end{itemize}
\end{multicols}

\noindent Our model outperforms previous models on the \textsc{Cresci-2018} dataset (table~\ref{tab:results}). We perform an ablation study to see the impact of different neighborhood features we have introduced.

\begin{table}[t]
\centering
\begin{tabular}{@{}lcccccc@{}}
\toprule
\textbf{Model}           & \textbf{TPR} & \textbf{FPR}  & \textbf{Acc} & \textbf{F-score} & \textbf{Recall@p95} & \textbf{AUC} \\ \midrule
Different model & 0.50 & 0.50  & 0.50 & 0.50     & 0.50        & 0.50 \\
SVM             & 0.50 & 0.50  & 0.50 & 0.50     & 0.50        & 0.50 \\
Random forest   & 0.50 & 0.50  & 0.50 & 0.50     & 0.50        & 0.50 \\
NN              & 0.50 & 0.50  & 0.50 & 0.50     & 0.50        & 0.50 \\
NN + GF         & 0.50 & 0.50  & 0.50 & 0.50     & 0.50        & 0.50 \\
NN + GF + NF    & \textbf{0.90} & \textbf{0.01} & \textbf{0.90} & \textbf{0.90}     & \textbf{0.90}        & \textbf{0.9} \\ \bottomrule
\end{tabular}
\caption{Placeholder evaluation results}
\label{tab:results}
\end{table}